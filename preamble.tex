

%\usepackage[german]{babel}             % Language german; replace ngerman by english when required
\usepackage[T1]{fontenc}                % For better wort separation
\usepackage[utf8]{inputenc}             % For äöüß etc.
\usepackage{lmodern}                    % Better font resolution
\usepackage{amsthm}                     % For theorems, definitions, propositions etc.
\usepackage{amsmath}                    % Makes everything better
\usepackage{mathtools}
\usepackage{comment}                    % For commenting out blocks of text
\usepackage{tikz}                       % Tikz ist kein Zeichenprogramm
\usepackage{stmaryrd}			% For \longarrownot\longrightarrow
\usepackage{array}
\usepackage{graphicx}
\usepackage{centernot}

\usetikzlibrary{arrows}
\usetikzlibrary{calc}

%%%%%%%%%%%%%%%%%%%  theorem environment %%%%%%%%%%%%%%%%%%%

%\theoremstyle{definition}
%\newtheorem{definition}{Definition}[section]
%\newtheorem{example}[definition]{Example}

%\theoremstyle{plain}
%\newtheorem{corollary}[definition]{Corollary}
%\newtheorem{theorem}[definition]{Theorem}
%\newtheorem{lemma}[definition]{Lemma}

%\theoremstyle{remark}
%\newtheorem{remark}[definition]{Remark}

%\theoremstyle{definition}
%\newtheorem{definition}{Definition}
%\newtheorem{theorem}{Theorem}[section]
%\newtheorem{proposition}[thm]{Proposition}
%\newtheorem{remark}[thm]{Remark}

%\newtheorem{lemma}[thm]{Lemma}
%\newtheorem{corollary}[thm]{Corollary}

\setbeamertemplate{theorem}[ams style]
\setbeamertemplate{theorems}[numbered]
\newtheorem{proposition}[theorem]{Proposition}

\theoremstyle{definition}
\newtheorem{remark}[theorem]{Remark}

\theoremstyle{example}
\newtheorem*{example*}{Example}


%%%%%%%%%%%%%%%%%%%   operatornames %%%%%%%%%%%%%%%%%%%%%%%%
% Some shortcuts


% Some Shortcuts
\newcommand{\Nat}{\mathbb{N}}
\newcommand{\Integ}{\mathbb{Z}}
\newcommand{\Real}{\mathbb{R}}
\newcommand{\Rat}{\mathbb{Q}}
\newcommand{\Comp}{\mathbb{C}}
\newcommand{\terms}{\mathbb{T}}
\newcommand{\field}{\mathbb{F}}
\newcommand{\im}{\operatorname{Im}}     	
\newcommand{\Ker}{\operatorname{Ker}}
\newcommand{\Hom}{\operatorname{Hom}}
\newcommand{\Supp}{\operatorname{Supp}}
\newcommand{\Ann}{\operatorname{Ann}}
\newcommand{\HF}{\operatorname{HF}}
\newcommand{\HP}{\operatorname{HP}}
\newcommand{\HS}{\operatorname{HS}}
\newcommand{\HN}{\operatorname{HN}}
\newcommand{\hn}{\operatorname{hn}}
\newcommand{\mult}{\operatorname{mult}}
\newcommand{\LT}{\operatorname{LT}}
\newcommand{\LM}{\operatorname{LM}}
\newcommand{\IG}{\mathcal{I}_G}
\newcommand{\MS}{\operatorname{MS}}
\newcommand{\LC}{\operatorname{LC}}
\newcommand{\GL}{\operatorname{GL}}
\newcommand{\NR}{\operatorname{NR}}
\newcommand{\id}{\operatorname{id}}
\newcommand{\End}{\operatorname{End}}
\newcommand{\ord}{\operatorname{ord}}
\newcommand{\Rel}{{\operatorname{Rel}}}
\newcommand{\Mat}{{\operatorname{Mat}}}
\newcommand{\trace}{\operatorname{trace}}
\renewcommand{\det}{\operatorname{det}}
\newcommand{\characteristic}{\operatorname{char}}
\newcommand{\Rad}{\operatorname{Rad}}
\newcommand{\rey}{\varrho}
\newcommand{\isom}{\cong}
\newcommand{\calA}{{\mathcal{A}}}
\newcommand{\calB}{{\mathcal{B}}}
\newcommand{\calC}{{\mathcal{C}}}
\newcommand{\calI}{{\mathcal{I}}}
\newcommand{\calZ}{{\mathcal{Z}}}
\newcommand{\tr}{^\operatorname{tr}}
\newcommand{\rewrite}[1]{{\stackrel{#1}{\longrightarrow}_s}}
\newcommand{\rewriteequiv}[1]{{\stackrel{#1}{\longleftrightarrow}_s}}
\newcommand{\reductionstep}[1]{{\stackrel{#1}{\longrightarrow}_{ss}}}
\newcommand{\colour}[1]{\color{#1}}
\newcommand{\bbar}[1]{{\mkern 4.0mu\overline{\mkern-4.0mu #1 \mkern-0.5mu}\mkern 0.5mu}}
\newcommand{\dotcup}{{\mathop{\dot\cup}}}



% Inserts an empty page
\newcommand{\emptypage}{\newpage\null\thispagestyle{empty}\newpage}

% For restricting a function to a certain subset - f|_A
\newcommand\restr[2]{{% we make the whole thing an ordinary symbol
		\left.\kern-\nulldelimiterspace % automatically resize the bar with \right
		#1 % the function
		\vphantom{\big|} % pretend it's a little taller at normal size
		\right|_{#2} % this is the delimiter
}}

% small matrices
\newcommand{\mat}[1]{\begin{psmallmatrix}#1\end{psmallmatrix}}

% font size
\newcommand{\size}[1]{\fontsize{#1}{0}\selectfont{}}


%%% Local Variables:
%%% mode: latex
%%% TeX-master: "main"
%%% End:
